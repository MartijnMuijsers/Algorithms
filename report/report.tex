\documentclass[11pt]{article}

\usepackage{amsmath,amssymb}
\usepackage{a4wide}
\usepackage{graphicx}
\usepackage{tikz}
\usepackage{algorithm}
\usepackage{algorithmic}
\usepackage{parskip}

\newcommand{\maxsize}[1]{\begin{quotation} {\sl \noindent Maximum size: #1.} \end{quotation}}

\newcommand{\crp}[1]{\begin{quotation} {\sl \noindent For the curve- and network-reconstruction problem: #1} \end{quotation}}

\newcommand{\example}[1]{\begin{quotation} {\sl \noindent Example: #1} \end{quotation}}

%%
% Theorem-Like Environments
%
\newtheorem{defin}{Definition}
  \newenvironment{mydefinition}{\begin{defin} \sl}{\end{defin}}
\newtheorem{theo}[defin]{Theorem}
  \newenvironment{mytheorem}{\begin{theo} \sl}{\end{theo}}
\newtheorem{lem}[defin]{Lemma}
  \newenvironment{lemma}{\begin{lem} \sl}{\end{lem}}
\newtheorem{coro}[defin]{Corollary}
  \newenvironment{corollary}{\begin{coro} \sl}{\end{coro}}
\newtheorem{obse}[defin]{Observation}
  \newenvironment{observation}{\begin{obse} \sl}{\end{obse}}

\newenvironment{proof}{\emph{Proof.}}{\hfill $\Box$ \medskip\\}

% TODO(robwu): Choose a more descriptive title
\title{2D curve and network reconstruction}
\author{
A. van den Boogaart \and
W. Brouwer \and
C. Mens \and
M. Muijsers \and
R. Wu
}
\date{\today}

\begin{document}

\newpage

\maketitle

\begin{abstract}
In this study  two formulations of solutions to effectively solve the problem of connecting a set of nodes in the 2D plane in an aesthetically pleasing manner, and one formulation of a solution for reconstructing a road network from a set of nodes in the 2D plane are proposed.
The first proposed solution reconstructs these nodes into a single curve.
The second proposed solution will yield a similar result as the first solution, but will distinguish multiple curves from one another.
The third proposed solution attempts to create a road network from nodes. This problem is distinct from the other two, as here the challenge lies in creating intersections, instead of avoiding them.

\end{abstract}

\section{Introduction}
\label{se:introduction}
The to be presented algorithms will solve the problem of reconstructing curves for a given set of points in the 2D plane. This problem has three slightly different forms: The first two require the reconstruction of a set of points into curves, in the first case there is only one curve, whereas in the second case there are multiple. The third part of the problem requires reconstructing to a road network.

The first two curve reconstruction algorithms are subject to some constraints: Curve segments should not intersect each other and all points in the input must be used in the curve.

The goal of the road network algorithm is to create a reasonable representation of the road network, such that a person would agree with the representation that the algorithm gives. This algorithm could be used to create a road map after data has been collected using the GPS locations of cars.

As with the other algorithms, there are some constraints: All points need to have at least one possible path to every other point and lines cannot intersect one another. However, unlike the other problems, points can be added to alleviate intersecting lines. All nodes are points in a 2D plane. These nodes are generated in advance. In practice, this can be done by laser scanners, or in the case of the network, by following roads and giving the location at certain intervals. 

There are multiple known solutions to these problems. One of these is `The Crust Algorithm' \cite{crust}, which first creates a Voronoi diagram, then uses Delaunay triangulation between the Voronoi diagram and the Voronoi vertices. The Crust Algorithm works well in three dimensions, but has difficulty with detecting sharp corners which are common in two dimensional reconstruction.
A different algorithm, specifically made for two dimensional reconstruction is `Gathan' \cite{gathan}. This provides a good solution for general cases, but does not allow for selecting multiple or single curves and relies on difficult mathematical graphs.
DISCUR \cite{discur} is another algorithm solving the curve reconstruction problem. It has no parameters, but requires dense sampling points to recontruct correctly.
For the network problem there exists a road network reconstruction algorithm \cite{chen} which obtains subcurves by creating a voronoi diagram.

Delaunay triangulation \cite{delaunay} could possibly be used to create a relatively small set of edges for the third problem, after which a rectilinear spanning graph or straight lines can be determined between points. This will improve efficiency, as it has a running time of $O(n \log{n})$, whereas Kruskal's algorithm will check all possible edges initially, resulting in $O(n^2 \log{n})$.

For the single curve reconstruction problem, we have also looked at the convex hull \cite{convex}. A convex hull will create segments between all the outermost points, such that no unconnected point is outside of the enclosed area. Like putting a rubber band around all the points. This is useful for the outermost nodes, but does not do anything with the nodes inside this outline.

For solving these three problems, we present an algorithm for each problem.
To solve the single curve reconstruction problem, the algorithm creates a convex hull and then continues by removing the largest edge and then reconnecting the figure by creating 2 new segments with another nearby point.
Multiple curves are solved using a seperate algorithm, which first connects the shortest segments such that there are no intersecting lines, after which it will seek out sharp angles and tries to remove and replace these with obtuse angles.
The road network reconstruction problems is solved using a minimum spanning tree. After reconstructing this tree, the algorithm finds nodes with degree 1 and connects these to a nearby node. To correctly create intersections, nodes with degree 3 or higher will check their surrounding segments and will add a point based on their angles.

\begin{figure}[ht!]
\centering
\includegraphics[scale=0.3]{images/outputOverview.png}
\caption{Output of the single curve (Left), multiple curve (Middle) and Network (Right) reconstruction algorithms.}
\label{outputOverview}
\end{figure}$ $\\


\section{The algorithms}
\label{se:algorithms}
\subsubsection{Background definitions}
Let $U = [0,1]^2$ be a set of points in a two-dimensional space.
For any $u,v \in U$, the line segment between $u$ and $v$ is denoted as $s_{u,v}$. $u$ and $v$ are called the endpoints of $s_{u,v}$. Each of the presented algorithms take $P \subset U$ as input, and output $S \subset \{s_{p,q} | p,q \in P \land p \neq q \}$.
 Two distinct points $a$ and $b$ are called \textit{connected} if there is a segment in $S$ whose endpoints are $a$ and $b$.
 Two segments are \textit{intersecting} each other if the line segments have a common point besides than the endpoints.
%TODO (rob): The next line does not belong here, it's a detail specific to the algorithm, right?
%None of these output line segments intersect another line segment in $S$.

For any $p,q \in U$, $d(p,q)$ is the Euclidean distance between $p$ and $q$. %\ref{todo:euclid_distance}.

For every $p \in P$ and $n \in \mathbb{N}$, $Adj_{p,n}$ is an adjacency list consisting of the $n$ nearest points to $p$, sorted in ascending order.

$\varphi(s_{u,v}, s_{v_w})$ is the absolute value of the smallest angle between two segments that share a common point, in degrees.

For any $p \in P$, the degree of $p$ denoted by $deg(p)$ is the number of line segments that are connected to $p$.

\subsection{Single curve reconstruction}
\subsection{Multiple curve reconstruction}
\subsection{Network Algorithm}

\section{Experimental evaluation}
\subsection{Generating input}
To properly test the presented algorithms, many test cases have been created. The majority of these test cases were created by hand, made to be as difficult as possible for the algorithms. Some have sharp corners, or have points with large gaps between them as shown in figure \ref{single}.

\begin{figure}[ht!]
\centering
\includegraphics[scale=0.2]{anglesamplerate.png}
\caption{Test case with low sampling and a sharp angle (left) and the resulting output from the single curve algorithm (right.)}
\label{single}
\end{figure}

Other test cases have been created to test the running time of the algorithm, with the amount of nodes ranging from $100$ to $10000$. This same input is then used on all algorithms.

\begin{table}[ht!]
    \begin{tabular}{lrrrrr}
    ~                       & 100 nodes & 500 nodes & 1.000 nodes & 5.000 			nodes & 10.000 nodes \\
    Single Reconstruction   & 33ms      & 126ms     & 782ms       & 3.985ms     	& 20.604ms     \\
    Multiple Reconstruction & 38ms      & 146ms     & 207ms       & 2.947ms     	& 7.387ms      \\
    Network Reconstruction  & 18ms      & 85ms      & 143ms       & 1.996ms     	& 4.802ms      \\
    \end{tabular}
\caption{Table showing the computing time required for different input sizes.}
\end{table}

The multiple curve algorithm has additional test cases that are not run by the single curve algorithm, made specifically to test whether it correctly distinguishes between the multiple curves. These include curves enclosed in another curve, some curves that are close to each other and multiple curves that spiral around each other, some examples of these test cases are given in figure \ref{multi}.

\begin{figure}[ht!]
\centering
\includegraphics[scale=0.3]{multiInput.png}
\caption{A collection of test cases specifically for the multiple curve algorithm, with the correct output displayed to the right of it.}
\label{multi}
\end{figure}

The network algorithm has different input than the other two given curve reconstruction algorithms, so it therefore also requires specific test cases. For this, test cases have been modeled after real life road maps. Some test have also been created to test specific parts of the algorithm, for instance a inputs that has close parallel lines, merging roads and roundabouts.

\begin{figure}[ht!]
\centering
\includegraphics[scale=0.3]{networkInput.png}
\caption{A large intersection to test for parrallel and merging lines (Left) Simple neighbourhood (Right)}
\label{network}
\end{figure}$ $\\

\subsection{Resulting output}
Dingen die je erin kan zetten:\\
Parameters, input sizes, sampling conditions, wat niet werkt en waarom niet, allemaal met plaatjes/tabellen erbij. Geef ook aan hoe je de werking van het algoritme kan zien in de output.

\subsubsection{Single curve}

\subsubsection{Multiple curves}
\subsubsection{Network reconstruction}

\subsection{Conclusion}

\section{Concluding remarks}


\subsection{Future Work}
\bibliographystyle{plain}

\begin{thebibliography}{50}

\bibitem{crust}

Amenta, Nina, Marshall Bern, and Manolis Kamvysselis. 
"A new Voronoi-based surface reconstruction algorithm." 
\textit{Proceedings of the 25th annual conference on Computer graphics and interactive techniques}, 1998.

\bibitem{kruskal}
J.B. Kruskal.
On the shortest spanning subtree of a graph and the traveling salesman problem.
In \emph{Proceedings of the American Mathematical Society},7: 48-50, 1956.

\bibitem{chen}
D. Chen, L.J. Guibas, J. Hershberger, J. Sun.
Road Network Reconstruction for Organizing Paths.
In \emph{Proceedings  of  21st  ACM-SIAM  Symposium  on  Discrete  Algorithms}, 10: 1309-1320, 2010.

\bibitem{discur}
Zeng, Yong, et al. "A distance-based parameter free algorithm for curve reconstruction." \textit{Computer-Aided Design} 40.2: 210-222, 2008 .

\bibitem{gathan}
Dey, Tamal K., and Rephael Wenger. "Reconstruction curves with sharp corners." \textit{Proceedings of the sixteenth annual symposium on Computational geometry.} ACM, 2000.

\bibitem{delaunay}

\bibitem{convex}


\end{thebibliography}







\end{document}

