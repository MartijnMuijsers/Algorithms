\documentclass[11pt]{article}

\usepackage{amsmath,amssymb}
\usepackage{a4wide}
\usepackage{graphicx}

\newcommand{\maxsize}[1]{\begin{quotation} {\sl \noindent Maximum size: #1.} \end{quotation}}

\newcommand{\crp}[1]{\begin{quotation} {\sl \noindent For the curve- and network-reconstruction problem: #1} \end{quotation}}

\newcommand{\example}[1]{\begin{quotation} {\sl \noindent Example: #1} \end{quotation}}

%%
% Theorem-Like Environments
%
\newtheorem{defin}{Definition}
  \newenvironment{mydefinition}{\begin{defin} \sl}{\end{defin}}
\newtheorem{theo}[defin]{Theorem}
  \newenvironment{mytheorem}{\begin{theo} \sl}{\end{theo}}
\newtheorem{lem}[defin]{Lemma}
  \newenvironment{lemma}{\begin{lem} \sl}{\end{lem}}
\newtheorem{coro}[defin]{Corollary}
  \newenvironment{corollary}{\begin{coro} \sl}{\end{coro}}
\newtheorem{obse}[defin]{Observation}
  \newenvironment{observation}{\begin{obse} \sl}{\end{obse}}

\newenvironment{proof}{\emph{Proof.}}{\hfill $\Box$ \medskip\\}

% TODO(robwu): Choose a more descriptive title
\title{2D curve and network reconstruction}
\author{
A. van den Boogaart \and
W. Brouwer \and
C. Mens \and
M. Muijsers \and
R. Wu
}
\date{\today}

\begin{document}

\newpage

\maketitle

\begin{abstract}
We present three algorithms that connect all nodes in an unorganised set in the 2D plane in an aestetically pleasing manner. 
The first algorithm will reconstruct a set of nodes into a single curve, such that no lines intersect. 
The second algorithm will create multiple curves from the set of nodes, also in such a way that no lines intersect. 
The third and final algorithm will attempt to create a road network from the nodes, and can add additional nodes when intersections occur.\\

\end{abstract}



\section{Introduction}
\label{se:introduction}
\maxsize{2 pages}
The problem that we attempt to solve is that of reconstructing a set of points back into curves. 
This problem is split into three different parts. The first two require the reconstruction into curves, in the first case there is only one curve, whereas in the second case there are mutiple. 
This means that in the second case it is also required to distinguish multiple different curves from each other.\\
All nodes are points in a 2D plane, given by floating points ranging from 0 to 1. 
The nodes are not determined randomly, but are determined beforehand. In practice, these datasets can be created by laserscanners, or in the case of the network, by following roads and giving the location at specific intervals.\\
There are multiple known solutions to these problems. One of these is known as 'The Crust Algorithm'[1], which first creates a Volonoi diagram, then uses Delaunay triangulation between the Volonoi diagram and the Volonoi vertices.\\
The Crust Algorithm solves the first problem that is described above, but needs additional code to determine the multiple curve problem.\\


The introduction usually starts with a description of and
a motivation for the general problem area.
%
\crp{You would start by
describing the general problem of object reconstruction from point data,
mention how data can be obtained (laser scanners), discuss network reconstruction, etcetera.}
%
After introducing the general problem area, you zoom in to the specific problem
studied in the paper. I like to already discuss previous work here. This way
you can explain where the specific problem fits into the state-of-the-art
and why it is interesting. Ideally, the discussion of the previous work
culminates in a clear statement about what is still missing in the current
state-of-the-art: namely an answer to the specific problem you study.
%
\crp{You would now discuss the specific variants that you study, 
what is known about them, which general reconstruction methods may or may not apply, and so on.}
%
Then you give an overview of your results.
%
\crp{You could give a high-level description of the
approach(es) you have used and relate them to approaches found in the literature.
State the theoretical guarantees (on running time,
for instance, or on other aspects) that you may have proved for your algorithms
and mention the main conclusions from the experiments.}
%
Finally, you can give an overview of the structure of the rest of your paper.
Personally, however, I do not find these overviews very useful: I prefer to
integrate this with the previous part of the introduction, where the overview
of the results is given.

\section{The algorithms}
\label{se:algorithms}
\maxsize{8 pages. Use subsections where appropriate.}
%
The description of the algorithms should be such that a programmer can
implement them without much difficulty. It is good practice to first explain the
main ideas behind the algorithm at a more intuitive level, and then give a
detailed description (for example using pseudo-code).
Don't forget to describe which supporting data structures you use:
linked lists, arrays, search trees, and so on. For standard data structures
from the literature you do not need to explain how they work; a reference
to the literature suffices.
%
\example{We store the set $P$ of points in a red-black tree~\cite{clrs-ia-01},
         using the $x$-coordinate of each point as its key.}
%
Try to theoretically analyze the worst-case running time of your algorithms
and the amount of storage they use. Also try to say something about the quality of your algorithm: you might
be able to prove that the algorithm is guaranteed to find the correct solution if the
input has certain well-defined properties, or you might be able to give examples
of inputs for which the algorithm will fail to give a correct solution.


\section{Experimental evaluation}
\label{se:evaluation}
\maxsize{8 pages. Use subsections where appropriate.}
Describe the experiments (for example how you generated the input and on which machine you
ran the experiments), give the results of the experiments (in the form of tables or graphs),
and discuss the results. Here you can also include pictures of your output for a few tests.
Relate the outcome of the experiments to your theoretical analysis.

Note that the experimental data in itself are not the main result: the conclusions
you can draw are what makes the data interesting.





\section{Concluding remarks}
\label{se:conclusions}
\maxsize{1 page}
Give a short overview of the main results, discussing both the strong
points of your algorithms as well as their weak points, and ideas for improvements
(``future work'').


\bibliographystyle{plain}

\begin{thebibliography}{}

\bibitem{} Here you put your references. See the next references for examples, and see below
for guidelines on how to do this.

\bibitem{a-raoa-02}
S. Albers.
On randomized online scheduling.
In \emph{Proc. 34th ACM Sympos. Theory Comput.}, pages 134--143, 2002.

\bibitem{clrs-ia-01}
T.H. Cormen, C.E. Leiserson, R.L. Rivest and C. Stein.
\emph{Introduction to Algorithms} (2nd edition).
MIT Press, 2001.

\bibitem{m-apca-83}
N. Megiddo.
Applying parallel computation algorithms in the design of serial algorithms.
\emph{J. ACM} 30: 852--865 (1983).

\end{thebibliography}

Make sure your references are well-polished and complete.
References are typically to books and to papers in scientific
journals and conference proceedings. References to web pages are hardly
ever appropriate.
The list of references is usually ordered alphabetically by first author (although some journals
list them in the order they are cited for the first time in the paper).
Only put in references that are actually cited in your paper.
Formatting can be done as in the example above---note that these references have
nothing to do with the topic of the DBL-project---, following the following rules:
\begin{itemize}
\item for journals:
      Authors. Title of paper. \emph{Journal Name (italic)} volume: page numbers (year).
      See reference~\cite{m-apca-83}.
\item for conference proceedings:
      Authors. Title of paper. In \emph{Proc. Conference Name and number (italic)}, pages xxx--yyy, year.
      See reference~\cite{a-raoa-02}
\item for books: Authors. \emph{Book title (italic)}. publisher, year.
      See reference~\cite{clrs-ia-01}
\end{itemize}
Note that names of journals and conferences are usually abbreviated. There is a more
or less standard way of doing this (for example, \emph{J. ACM} is stands for \emph{Journal of the ACM},
but how you do it exactly is not so important,
as long as you are consistent and list all the necessary information.






\end{document}

